%%%%%%%%%%%%%%%%%%%%%%%%%%%%%%%%%%%%%%%%%
% Homework Assignment Article
% LaTeX Template
% Version 1.3.1 (ECL) (08/08/17)
%
% This template has been downloaded from:
% Overleaf
%
% Original author:
% Victor Zimmermann (zimmermann@cl.uni-heidelberg.de)
%
% License:
% CC BY-SA 4.0 (https://creativecommons.org/licenses/by-sa/4.0/)
%
%%%%%%%%%%%%%%%%%%%%%%%%%%%%%%%%%%%%%%%%%

%----------------------------------------------------------------------------------------

\documentclass[a4paper]{article} % Uses article class in A4 format

%----------------------------------------------------------------------------------------
%	FORMATTING
%----------------------------------------------------------------------------------------

\addtolength{\hoffset}{-2.25cm}
\addtolength{\textwidth}{4.5cm}
\addtolength{\voffset}{-3.25cm}
\addtolength{\textheight}{5cm}
\setlength{\parskip}{0pt}
\setlength{\parindent}{0in}

%----------------------------------------------------------------------------------------
%	PACKAGES AND OTHER DOCUMENT CONFIGURATIONS
%----------------------------------------------------------------------------------------

\usepackage{blindtext} % Package to generate dummy text
% \usepackage[style=numeric,sorting=none]{biblatex}
\usepackage{charter} % Use the Charter font
\usepackage[utf8]{inputenc} % Use UTF-8 encoding
\usepackage{microtype} % Slightly tweak font spacing for aesthetics

\usepackage[english]{babel} % Language hyphenation and typographical rules

\usepackage{amsthm, amsmath, amssymb} % Mathematical typesetting
\usepackage{float} % Improved interface for floating objects
\usepackage[final, colorlinks = true, 
            linkcolor = black, 
            citecolor = black]{hyperref} % For hyperlinks in the PDF
\usepackage{graphicx, multicol} % Enhanced support for graphics
\usepackage{xcolor} % Driver-independent color extensions
\usepackage{marvosym, wasysym} % More symbols
\usepackage{rotating} % Rotation tools
\usepackage{censor} % Facilities for controlling restricted text
\usepackage{listings, style/lstlisting} % Environment for non-formatted code, !uses style file!
\usepackage{pseudocode} % Environment for specifying algorithms in a natural way
\usepackage{style/avm} % Environment for f-structures, !uses style file!
\usepackage{booktabs} % Enhances quality of tables

\usepackage{tikz-qtree} % Easy tree drawing tool
\tikzset{every tree node/.style={align=center,anchor=north},
         level distance=2cm} % Configuration for q-trees
\usepackage{style/btree} % Configuration for b-trees and b+-trees, !uses style file!

% \usepackage[backend=biber,style=numeric,
            % sorting=nyt]{biblatex} % Complete reimplementation of bibliographic facilities
% \addbibresource{ecl.bib}
\usepackage{csquotes} % Context sensitive quotation facilities

\usepackage[yyyymmdd]{datetime} % Uses YEAR-MONTH-DAY format for dates
\renewcommand{\dateseparator}{-} % Sets dateseparator to '-'

\usepackage{fancyhdr} % Headers and footers
\pagestyle{fancy} % All pages have headers and footers
\fancyhead{}\renewcommand{\headrulewidth}{0pt} % Blank out the default header
\fancyfoot[L]{School of Computing, Macquarie University} % Custom footer text
\fancyfoot[C]{} % Custom footer text
\fancyfoot[R]{\thepage} % Custom footer text

\usepackage{comment}
\newcommand{\note}[1]{\marginpar{\scriptsize \textcolor{red}{#1}}} % Enables comments in red on margin

\usepackage{listings}
\usepackage{color}

\definecolor{dkgreen}{rgb}{0,0.6,0}
\definecolor{gray}{rgb}{0.5,0.5,0.5}
\definecolor{mauve}{rgb}{0.58,0,0.82}

\lstset{frame=tb,
  language=Java,
  aboveskip=3mm,
  belowskip=3mm,
  showstringspaces=false,
  columns=flexible,
  basicstyle={\small\ttfamily},
  numbers=none,
  numberstyle=\tiny\color{gray},
  keywordstyle=\color{blue},
  commentstyle=\color{dkgreen},
  stringstyle=\color{mauve},
  breaklines=true,
  breakatwhitespace=true,
  tabsize=3
}
%----------------------------------------------------------------------------------------

\begin{document}

%----------------------------------------------------------------------------------------
%	TITLE SECTION
%----------------------------------------------------------------------------------------

\title{COMP3100 project report} % Article title
\fancyhead[C]{}
\hrule \medskip % Upper rule
\begin{minipage}{1\textwidth} % Center of title section
\centering 
\large % Title text size
Project report: Stage 2\\ % Assignment title and number
COMP3100 Distributed Systems, S2, 2022\\
\normalsize % Subtitle text size
SID: 44901070, Name: Calleum Pecqueux
%%%%\\ % Assignment subtitle
\end{minipage}
\medskip\hrule % Lower rule
\bigskip

%----------------------------------------------------------------------------------------
%	ARTICLE CONTENTS
%----------------------------------------------------------------------------------------
\section{Introduction}
This project aims to implement a client for a client-server model discrete event
simulator. The server for this project (ds-server) is
provided~\cite{mq-dist-github}. The goals for this stage of the project include

In this report, I aim to outline the design and implementation of the client,
and review the design decisions made. 

The rest of this report is organised as follows. 

Section~\ref{sec:section2}
provides a high-level problem definition and description of the scheduling
problem and solution. 

Section~\ref{sec:section3} provides details of design decisions at the
architectural level, how these interact with the project considerations 
and constraints, and the functionality of the client-side simulator.

Section~\ref{sec:section4} outlines the choices made at the implementation
level, including walking through the control loop and algorithms used to enforce
the correctness of the client-server protocol.

References and a link to the code repository \cite{calleum-ds-sim} are included.

\section{System Overview }
\label{sec:section2}
The system implements a Largest-Round-Robin Scheduler following the protocol
defined by Lee in ds-sim user-guide \cite{mq-dist-github-userguide}.

\section{System Design}
\label{sec:section3}
The System is split into logical components - Job, Server, utility classes for
each, Client, ConnectionHandler and configuration classes. This follows the
common pattern used in java of having way too many classes, all with 10 lines
each.
\subsection{Client-server Communication}
\begin{verbatim}
        +---------------------------------------------------------+
        |              Socket: localhost:50000                    |
        +----------^-+--------------------------^---+-------------+
                   | |                          |   |
                   | |                          |   |
         +---------+-v-----+      +-------------+---v-------------+
         |                 |      |     sendMsg() recvMsg()       |
         |                 |      |                               |
         |                 |      |                               |
         |                 |      |                               |
         +-----------------+      +-------------------------------+
               Server                           Client
\end{verbatim}

\section{Client-side Simulator Implementation}
\label{sec:section4}
The main entry point to the program is the function run() called from the main
function of App.java: 

\begin{lstlisting}
public void run() {
        try {
            makeHandshake();
        } catch (Exception e) {
            LOG.info("Exception" + e);
        }

        while (isConnected()) {
            handleEvent(recvMsg());
        }
    }
\end{lstlisting}

The control loop is simply checking if the static class scoped variable
"connected" is still true, indicating no errors between the client and server.
Error handling naively closes the socket when an unknown behaviour occurs, where
the variable "connected" is flipped to false..

The client loop is entered after the client finishes the handshake with the
function makeHandshake();

\begin{lstlisting}
public void makeHandshake() throws IOException {
        LOG.info("Initiating Handshake");
        sendMsg(CmdConstants.HELO);
        final String msg = recvMsg();
        LOG.info(msg);
        final String auth = authMsg();
        LOG.info(auth);
        sendMsg(authMsg());
        this.connected = true;
    }
\end{lstlisting}
%----------------------------------------------------------------------------------------
%	REFERENCE LIST
%----------------------------------------------------------------------------------------
\bibliographystyle{ieeetr}
\bibliography{comp3100project}
% \printbibliography

%----------------------------------------------------------------------------------------

\end{document}
